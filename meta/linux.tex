\secrel{\class{MLinux}}\label{mlinux}\secdown

Популярная ОС общего назначения + ОС №1 для встраиваемых систем, включая мобильные телефоны на \andr,
SOHO роутеры, железо для хобби-автоматики (\rpi), ТВ-приставки, панели управления технологического оборудования, и т.д.

С некоторой адаптацией используется в стойках ЧПУ не только хоббийного класса, но и в панелях оператора профессионального
оборудования, типа Sinumeric 840D. \href{http://linuxcnc.org/}{LinuxCNC} обеспечивает полнофункциональную систему управления
для ЧПУ станков и роботов.

\bigskip
Для совместимости с \win\ --- ориентируемся на пакет \href{http://www.mingw.org/}{\mingw},
получаем возможность компиляции как в \win, так и кросс-компиляции с рабочей станции на \linux.

\clearpage
\secrel{\class{BR}: \emlin\ на основе \br}\label{br}

Хороший удобный дистрибутиво-конструктор, позволяет собрать собственный мини\linux\ для разнообразных
архитектур (x86/ARM/MIPS), включает множество пакетов, которые включаются в \emph{монолитную сборку \linux}.
Из минусов\ --- пакеты и внешние репозитории как у больших дистрибутивов и OpenWrt не поддерживаются.
Хорошо подходит для изготовления прошивок для разнообразных многофункциональных и специализированных устройств,
плохо\ --- в качестве ОС для всего что напоминает ПК общего назначения\note{в т.ч. на процессорах ARM\ --- нетбуки, \rpi,\ldots}.

\begin{itemize}
    \item \url{https://buildroot.org/}
    \item \href{https://habr.com/ru/post/448638/}{1. Общие сведения, сборка минимальной системы, настройка через меню}
    \item \href{https://habr.com/ru/post/449348/}{2. Создание конфигурации своей платы; применение external tree, rootfs-overlay, post-build скриптов}
\end{itemize}

\lst{meta/Linux.st}{\file{Linux.st}}

\secup
