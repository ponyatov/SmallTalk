\secrel{Метапрограммирование}\label{meta}\secdown

\term{Метапрограммирование}\ --- \textit{программы которые пишут программы}\\
вид программирования, связанный с созданием программ, 
которые \emph{порождают другие программы как результат своей работы}\ (в частности, на стадии компиляции их исходного кода), 
либо программ, которые меняют себя во время выполнения (самомодифицирующийся код).

\clearpage
\term{Генерация кода} позволяет получать программы при меньших затратах времени и 
усилий на кодирование, чем если бы программист писал их вручную целиком, 
самомодификация позволяет улучшить свойства кода (размер и быстродействие),
проводить оптимизации, используя статистику выполнения на реальных данных.

Нас прежде всего интересует (кросс-)\term{трансляция моделей}, описанных через исполняемые структуры данных \ref{eds}
на языке \st, с использованием всех фишек, предоставляемых \st-системой как IDE\note{гостевой операционной системой},
расширяемой пользователем.

Также стоит упомянуть что \st\ поддерживает метапрограммирование сам по себе. Используя системные классы,
вы не только можете сгенерировать программу в виде строки и запустить встроенный компилятор, но и 
работать на низком уровне непосредственно с байт-кодом.

% \clearpage
\subsecly{Подключение репозитория \met}

Ограниченная версия \met\ описанная в этом руководстве, поставляется в составе
файлов, доступных по адресу\\
\url{https://github.com/ponyatov/Smalltalk/tree/master/metaL}\\
или локально в каталоге \file{$\sim$/Smalltalk/metaL}.

\fig{img/metacello.png}{height=.3\textheight}

\noindent
создать пустой пакет \met, \menu{World>Tools>Monticello Browser}, выбрать в фильтре \met, добавить репозиторий \menu{+Repository>filetree://},
выбрать каталог \file{$\sim$/Smalltalk/metaL}. 
В правой части появится \file{.../metaL.package}. Сделать на нем \menu{\rms>Open package>metaL.package>Load}

\fig{img/opengit.png}{height=.45\textheight}

\fig{img/metopen.png}{height=.5\textheight}

При работе с Monticello система запросит ваше имя, которое будет вставляться при создании новых
версий в системе контроля версий:

\fig{img/Author.png}{width=.7\textwidth}

\clearpage
После установки пакета через Monticello или сброса в начальное состояние
после ручного ввода кода из этого руководства требуется переинициализация
через метод \verb|MetaL.install|.

\medskip
\fig{img/plinstall.png}{height=.5\textheight}

Этод \term{метод класса}\ вызовет соответствующие методы у всех классов,
наследованных от \class{MFrame}.

\fig{img/install.png}{width=.7\textwidth}



\clearpage
\secrel{\class{MFrame}: универсальный объект данных}\label{frame}\secdown

\term{Фрейм}\ --- универсальный объект, способный представлять любые данные,
в том числе и \emph{исполняемое\note{интерпретируемое движком \met}\ представление
программ} в виде \textit{направленного (гипер)графа}. От просто объекта отличается
наличием предопределенной структуры, и \term{унифицированного интерфейса} для
манипуляции данными:

\begin{description}

    \item{\var{type}} (константа класса)\\ признак типа/класса, в терминах \met: строчные буквы,
    отсутствие префикса M, например для базового фрейма \verb|type=frame|

    \item{\var{val}}\ ue\\ скалярное значение: \textit{имя фрейма}, строка, число,..

\end{description}
\begin{description}[nosep]

    \item{\var{slot}}\verb|{}s|\\ \term{слот}ы = \term{атрибут}ы, ассоциативный массив в каждом объекте
    хранит ссылки на другие фреймы, привязанные через строковой ключ;\\
    в графическом представлении связки отображаются через именованные ребра графа\\
    слоты могут рассматриваться как переменные, локальные для фрейма, именованные ссылки,
    \term{среда} (environment), словарь (в языке Форт)\\
    \textit{любой объект может быть использован как словарь} \ref{dict}

    \item{\var{nest}}\verb|[]ed|\\ вложенные элементы = упорядоченный контейнер = массив = стек\\
    возможность упорядоченного хранения была добавлена к оригинальной концепции фрейма \cite{minsky}
    для представления \term{AST: абстрактных синтаксических деревьев} и графов\ --- общеизвестная структура данных,
    используемая при разработке трансляторов и компиляторов большинства языков программирования. 
    
\end{description}

Фрейм является базовым элементом в модели \met, направленный граф формируется через ссылки на другие фреймы,
и может также быть использован в базах знаний, KR\&R\ --- Knowledge Rep\-resen\-tation and Rea\-so\-ning \cite{minsky}.
Их практических соображений, для человека наиболее удобной формой представления информации являются именно графы (диаграммы),
вы их можете увидеть практически на любой доске для совещаний: типы узлов выделены как разные графические изображения,
ребра и замкнутые контуры задают \term{отношения} между объектами, часто для элементов диаграмм заданы \term{атрибуты} в виде
различных надписей и цветового выделения.

\secup


\clearpage
\secrel{метод EDS: исполняемые структуры данных}\label{eds}

\term{EDS}: Executable Data Structure\ --- метод програмирования, который позволяет обойти
ограничения языка програмирования, который вы используете, за счет \emph{интерпретации структур данных}
как программ.

\term{Гомоикон\'{и}чность} (гомоиконность, англ. homoiconicity, homoiconic)\ --- 
свойство некоторых языков программирования, в которых структура программы похожа 
на его синтаксис, и поэтому внутреннее представление программы имеет такую же структуру, как её 
исходный код (то есть AST и синтаксис являются изоморфными).

Применение EDS позволяет не только затащить в любой язык произвольную парадигму программирования,
модель в терминах которой программист может формулировать обработку данных, но и некоторые мощные
вещи доступные только в гомоиконичных языках\note{Lisp, Scheme}: метапрограммирование, самопреобразования программ
во время исполнения (в рантайме), автоматических синтез кода по исходным данным, и т.п.
В результате мы получаем гибкость в подходах и форматах хранения программ, ценой определенной потери
вычислительной эффективности и памяти.

В качестве примера можно привести IoT\note{Интернет вещей}, в котором применяются очень маломощные
компьютеры типа микроконтроллеров Cortex-M0 с не более чем 20К ОЗУ, и десятками Кб памяти программ.
Если нам требуется обеспечить замену программного обеспечения, можно вместо прошивки в машинном коде,
обычно написанной на \ci, применить \term{интерпретацию байт-кода}, и стековую виртуальную машину.
При правильном выборе формата команд ВМ код программ оказыватся значительно компактнее машинного кода,
и мы можем не только съэкономить на объеме передаваемой прошивки, но получить дополнительный
профит в виде более стабильной работы системы (изоляция аппаратуры от программ пользователей, отсутствие
возможности "окрипичивания" при загрузке байт-кода в ОЗУ, контроль доступа к ресурсам,
конкурентное выполнение и т.п.).

В отличие от байткода\note{традиционное внутреннее представление программ в большинстве интерпретаторов,
в том числе и Java, и сам \st}, при применении EDS необязательно использовать специализированный
компилятор\ --- если язык X на котором вы пишете \term{EDS-интерпретатор} достаточно гибок, вы можете
описать программу на X, и вкомпилировать ее в исполняемый файл.

Стоит заметить, что EDS не предполагает существование исходного кода в виде текстовых файлов,
и реализацию какого-то специализированного языка программирования для их загрузки\ --- \textit{исполняемая
структура данных сама по себе является собственным "исходным кодом"}. \term{Парсер} это опция, и вы можете
ее использовать по собственному усмотрению, или наоборот полностью отказаться от текстового программирования,
и применять средства GUI, и самомодификацию программ.

Конкретную форму исполняемой структуры и архитектуру \term{виртуальной машины} вы тоже выбираете по собственному
усморению, в зависимости от требований, задачи, и предпочтений:
\begin{description}
    \item{\emph{байт-код}}\ классика, но с точки зрения методики EDS наихудший вариант: слишком низкоуровневое
    представление, фактически это машинный код ВМ, его неудобно генерировать, а больше всего\ --- проблемно
    модифицировать.
    \item{\emph{стек + словарь + массивы}}\ если некоторый язык позволяет работать с
    гетерогенными\note{хранить объекты разных или произвольных классов, например Python}
    структурами данных и хранить в них функции, на нем очень легко пишется интерпретатор языка Форт:
    программа этомассив данных и функций
    \item{\emph{вложенные и рекурсивные списки}}\ классика в квадрате, с легким привкусом антиквариата\ --- Lisp полностью
    определен как операции на исполняемых списках; недостаток: все есть список, поэтому другие структуры тоже приходится
    определять через список
    \item{\emph{AST, граф объектов}}\ самый высокоуровневый вариант, применяется
    нечасто\note{язык Пролог (?), и экспериментальные языки построенные на перезаписи графов} так как
    их интерпретация считается слишком медленной для вычислений и обработки данных,
    но если ориентироваться на метапрограммирование и преобразования программ, наоборот подходит больше всего
    \item{\emph{реляционная база данных}}\ ну, эээ... наверно возможно, но сложно представить как, а главное зачем
    (если не рассматривать сериализацию программ и хранимые процедуры интерпретирующие интерпретатор)
\end{description}

\secrel{Синтез кода на \ci/\cpp}\label{synth}\secdown

\fig{img/c89.png}{width=\textwidth}

\clearpage
Так как основное целевое направление \met\ это приложения для IoT и embedded систем,
главная цель\ --- синтезировать переносимый код на ANSI \ci\ и \cpp\ через \term{трансляцию
моделей} в \term{синтезируемое подмножество фреймов}. Под словом "синтезируемое"\ имеется в виду
набор классов, наследованных от \class{MFrame}$\rightarrow$\class{C89}$\rightarrow$\class{Cpp},
которые имеют набор методов\note{\term{интерфейс}} генерирующих код на соответствующем языке
в виде строк или текстовых файлов.

\bigskip
\emph{Автогенерация кода} решает основную проблему \st, которая не позволяет использовать его
в современных разработках\ --- обеспечение \term{интероперабельности} с существующими проектами,
API, и библиотеками.
Вторая проблема\ --- минимизация приложения (VM + файл образа).
Предлагается \textit{писать автоген} на \st\ со всеми фишками, но как результат получать
программы на любом другом языке программирования.

\clearpage
\secrel{Классы файлов автогенного исходного кода}\label{cfile}\label{cmakefile}

Для организации кода используется два файловых класса:
\begin{description}
    \item{\emph{\class{CFile}}} файлы .c .h .cpp .hpp (с общим синтаксисом)
    \item{\emph{\class{CMakefile}}} скрипты управления сборкой проекта
\end{description}

% \clearpage
\secrel{Примитивные (аппаратные) типы \ci}

\fig{dot/ctypes.png}{height=.7\textheight}

\secrel{\fn{main()}}

\lst{meta/main.st}{main.st}

\secup


\clearpage
\secrel{\class{MOS} спецификация операционных систем}\label{os}\secdown

Для генерации кода очень важной характеристикой является то, под какой \term{операционной системой}
будет запускаться наш сгенерированный код. Если посмотреть на современное состояние ИТ, 
тяжела и неказиста жизнь embedded программиста: 
\begin{description}%[nosep]
    \item{\emph{большая тройка} десктопа}: \win, \linux, \macos
    \item{\emph{серверные} (склонны к плесневению)}: Solaris, AIX, ВыньServer
    \item{\emph{embedded/RTOS}}: mbed, FreeRTOS, QNX,.., bare metal\note{голое железо без RTOS}
    \item{\emph{мобильные}}: \ios, \andr
    \item{великая и ужасная} \texttt{Жаба}
\end{description}

\clearpage
Если умножить количество ОС на количество поддерживаемых платформ \ref{HW}, получим \term{матрицу целевых
систем}\ минимум на полсотни вариантов. Если вы поддерживаете несколько сборок, и пару железок поставляемых вашей компанием, не проблема. Но если
нацеливаться на полноразмерный \term{гетерогенный}\ \iot\ со всей его кашей архитектур и непредсказуемого разнообразия железа\ldots

\lst{meta/OS.st}{\file{MOS.st}}

\secrel{\class{MLinux}}\label{mlinux}\secdown

Популярная ОС общего назначения + ОС №1 для встраиваемых систем, включая мобильные телефоны на \andr,
SOHO роутеры, железо для хобби-автоматики (\rpi), ТВ-приставки, панели управления технологического оборудования, и т.д.

С некоторой адаптацией используется в стойках ЧПУ не только хоббийного класса, но и в панелях оператора профессионального
оборудования, типа Sinumeric 840D. \href{http://linuxcnc.org/}{LinuxCNC} обеспечивает полнофункциональную систему управления
для ЧПУ станков и роботов.

\bigskip
Для совместимости с \win\ --- ориентируемся на пакет \href{http://www.mingw.org/}{\mingw},
получаем возможность компиляции как в \win, так и кросс-компиляции с рабочей станции на \linux.

\clearpage
\secrel{\class{BR}: \emlin\ на основе \br}\label{br}

Хороший удобный дистрибутиво-конструктор, позволяет собрать собственный мини\linux\ для разнообразных
архитектур (x86/ARM/MIPS), включает множество пакетов, которые включаются в \emph{монолитную сборку \linux}.
Из минусов\ --- пакеты и внешние репозитории как у больших дистрибутивов и OpenWrt не поддерживаются.
Хорошо подходит для изготовления прошивок для разнообразных многофункциональных и специализированных устройств,
плохо\ --- в качестве ОС для всего что напоминает ПК общего назначения\note{в т.ч. на процессорах ARM\ --- нетбуки, \rpi,\ldots}.

\begin{itemize}
    \item \url{https://buildroot.org/}
    \item \href{https://habr.com/ru/post/448638/}{1. Общие сведения, сборка минимальной системы, настройка через меню}
    \item \href{https://habr.com/ru/post/449348/}{2. Создание конфигурации своей платы; применение external tree, rootfs-overlay, post-build скриптов}
\end{itemize}

\lst{meta/Linux.st}{\file{Linux.st}}

\secup

\secrel{\class{MRTOS} модель ОС реального времени}\label{rtos}\secdown

Для приложений \iot\ необходима система многозадачности с поддержкой \term{жесткого реального времени}.
Для этого может быть использована одна из распространенных \term{ОСРВ} \ref{mbed},\ref{freertos}, или адаптивная
модель описанная средствами \met. Поддержка mainstream OS необходима для интеграции с существующими проектами,
и для упрощения поддержки\note{чтобы вам не приходилось читать коллегам лекции по метапрограммированию вместо работы над проектами}.

Модельная ОСРВ заинтересует исследователей, в потенциале может дать адаптивность и неограниченную гибкость в конфигрировании,
но для этого и сама модель, и средства вычислений на графах в \met\ должны быть достаточно зрелыми.
Интересна реализация \term{акторного микроядра} \ref{akka} обеспечивающего RT в пределах одного узла, и прозрачную кластеризацию в
распределенной вычислительной среде.

\secrel{\class{Mbed}}\label{mbed}

\secrel{\class{MFreeRTOS}}\label{freertos}

\secup


\secup


\clearpage
\secrel{\class{HW} аппаратные компоненты}\label{hw}\secdown

\lst{meta/HW.st}{\file{HW.st}}

\fig{dot/HW.png}{width=\textwidth}

\noindent
Поддерево классов \class{HW}\ предназначено в основном для описания аппаратных систем на микроконтроллерах:
включает минимальный набор компонентов, необходимых для любого \term{оконечного узла} \iot. Задача системы
\met\ --- взять описание заданного аппаратного модуля (\term{\class{HW}-модель}), и сгенерировать
пакет исходного кода на \ci, выполняющего\note{с учетом особенностей каждой конкретной железки и 
способа ее применения}:
\begin{itemize}
    \item инициализацию аппаратных компонентов
    \item обработку прерываний
    \item интерфейс аппаратных компонентов в акторной модели \ref{akka}
    \item увязать программную логику (модель прошивки) с "железом"
    \item реализовать слой ОС жесткого реального времени (\term{RTOS}) для акторов
          (планировщик, очереди сообщений с приоритетами, сервисы управления ресурсами) \ref{rtos}
\end{itemize}

\secup


\secup
