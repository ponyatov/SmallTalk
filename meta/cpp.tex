\secrel{Синтез кода на \ci/\cpp}\label{synth}\secdown

\fig{img/c89.png}{width=\textwidth}

\clearpage
Так как основное целевое направление \met\ это приложения для IoT и embedded систем,
главная цель\ --- синтезировать переносимый код на ANSI \ci\ и \cpp\ через \term{трансляцию
моделей} в \term{синтезируемое подмножество фреймов}. Под словом "синтезируемое"\ имеется в виду
набор классов, наследованных от \class{MFrame}$\rightarrow$\class{C89}$\rightarrow$\class{Cpp},
которые имеют набор методов\note{\term{интерфейс}} генерирующих код на соответствующем языке
в виде строк или текстовых файлов.

\bigskip
\emph{Автогенерация кода} решает основную проблему \st, которая не позволяет использовать его
в современных разработках\ --- обеспечение \term{интероперабельности} с существующими проектами,
API, и библиотеками.
Вторая проблема\ --- минимизация приложения (VM + файл образа).
Предлагается \textit{писать автоген} на \st\ со всеми фишками, но как результат получать
программы на любом другом языке программирования.

\clearpage
\secrel{Классы файлов автогенного исходного кода}\label{cfile}\label{cmakefile}

Для организации кода используется два файловых класса:
\begin{description}
    \item{\emph{\class{CFile}}} файлы .c .h .cpp .hpp (с общим синтаксисом)
    \item{\emph{\class{CMakefile}}} скрипты управления сборкой проекта
\end{description}

% \clearpage
\secrel{Примитивные (аппаратные) типы \ci}

\fig{dot/ctypes.png}{height=.7\textheight}

\secrel{\fn{main()}}

\lst{meta/main.st}{main.st}

\secup
