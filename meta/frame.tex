\clearpage
\secrel{\class{MFrame}: универсальный объект данных}\label{frame}\secdown

\term{Фрейм}\ --- универсальный объект, способный представлять любые данные,
в том числе и \emph{исполняемое\note{интерпретируемое движком \met}\ представление
программ} в виде \textit{направленного (гипер)графа}. От просто объекта отличается
наличием предопределенной структуры, и \term{унифицированного интерфейса} для
манипуляции данными:

\begin{description}

    \item{\var{type}} (константа класса)\\ признак типа/класса, в терминах \met: строчные буквы,
    отсутствие префикса M, например для базового фрейма \verb|type=frame|

    \item{\var{val}}\ ue\\ скалярное значение: \textit{имя фрейма}, строка, число,..

\end{description}
\begin{description}[nosep]

    \item{\var{slot}}\verb|{}s|\\ \term{слот}ы = \term{атрибут}ы, ассоциативный массив в каждом объекте
    хранит ссылки на другие фреймы, привязанные через строковой ключ;\\
    в графическом представлении связки отображаются через именованные ребра графа\\
    слоты могут рассматриваться как переменные, локальные для фрейма, именованные ссылки,
    \term{среда} (environment), словарь (в языке Форт)\\
    \textit{любой объект может быть использован как словарь} \ref{dict}

    \item{\var{nest}}\verb|[]ed|\\ вложенные элементы = упорядоченный контейнер = массив = стек\\
    возможность упорядоченного хранения была добавлена к оригинальной концепции фрейма \cite{minsky}
    для представления \term{AST: абстрактных синтаксических деревьев} и графов\ --- общеизвестная структура данных,
    используемая при разработке трансляторов и компиляторов большинства языков программирования. 
    
\end{description}

Фрейм является базовым элементом в модели \met, направленный граф формируется через ссылки на другие фреймы,
и может также быть использован в базах знаний, KR\&R\ --- Knowledge Rep\-resen\-tation and Rea\-so\-ning \cite{minsky}.
Их практических соображений, для человека наиболее удобной формой представления информации являются именно графы (диаграммы),
вы их можете увидеть практически на любой доске для совещаний: типы узлов выделены как разные графические изображения,
ребра и замкнутые контуры задают \term{отношения} между объектами, часто для элементов диаграмм заданы \term{атрибуты} в виде
различных надписей и цветового выделения.

\secup
