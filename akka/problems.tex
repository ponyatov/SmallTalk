\secrel{Проблемы акторной модели}\secdown

\begin{itemize}[nosep]
\item \url{https://habr.com/ru/post/324420/}
\item \url{https://habr.com/ru/post/324978/}
\end{itemize}

\secrel{Домен плохо ложится на акторную модель}

Бонусы акторной модели доступны только если задача хорошо ложится на модель
\emph{асинхронной односторонней} передачи сообщений.

\secrel{Перегрузка при отсутствии backpressure}

Если актор не успевает обрабатывать входящий поток сообщений, возникает неконтрлируемый рост
очереди на обработку. При асинхронной односторонней передаче сообщений очень проблемно
организовать механизм \term{backpressure}\ --- нотификация или приостановление работы акторов
с отправляющей стороны.

Другая сложность работы в режиме перегрузки\ --- нужно затачивать защиту от перегрузке под
конкретную задачу: выбравать старые либо новые сообщения, изменять обработку, перенаправлять
поток в хранилище пока обработчик не освободится.

Как одно из решений можно применять поллинг со стороны приемника, извещающий сендеры
о готовности принимать данные, или степень загрузки очереди, чтобы сендеры могли
приостановить свою работу.

\secrel{Доставка сообщений ненадежна}

Проблема естественна, никто не гарантирует доставку и сохранение порядка сообщений.
\begin{itemize}[nosep]
\item Перепосылка сообщения после таймаута
\item Откат операции
\end{itemize}

\secrel{Слишком часто нужна синхронность}

Акторная модель чисто асинхронная, практические задачи очень часто требуют двусторонний обмен
sender/receiver, привычка программировать на синхронных вызовах делает проблему критической.

\secrel{Обеспечение прозрачной распределенности по TCP/IP}

Усложнение протоколов синхронизации, (де)сериализации, и использования распространенных
протоколов сторонних по отношению к акторной системе.

\secup
