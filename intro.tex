% \clearpage
\secly{О книге}

Это заметки по использованию современного \st\ --- системы \ph. В основном это смесь переводов нескольких учебников
на русский язык. Основная цель\ --- показать метод метапрограммирования \ref{meta}\ через кодогенерацию,
позволяющий \emph{писать на \st\ \textit{на любом языке}}.

В каком-то смысле это ответ на систему \url{https://flprog.ru}\ тоже написанную на \st, но не следующую ее духу:
\emph{приложение для \st\ это \textit{расширение системы}}, а не изолированный пакет\ --- пользователю предоставляется доступ к расширению
как к библиотеке, с полным доступом и к ее исходному коду\note{поэтому приоритетно лицензирование под OpenSource + платныя поддержка},
и к базовой системе. \fl\ --- система для графической разработки прошивок для домашней и хоббийной автоматики на платформе Одурино.
Она поставляется бесплатно, но как закрытый standalone пакет для платформы \win\note{на сайте есть сборка под \linux,\ но
пока есть проблемы со шрифтами, совместима с ограниченным количеством систем, не поддерживаются архитектуры ARM и MIPS (Байкал-Т1)}.
В результате пользователи не могут адаптировать \fl\ под
свои задачи, и разработка завязана только на одного человека\note{vendor lock и в каком-то смысле bus factor\ --- достаточно чтобы
\href{https://www.youtube.com/channel/UCI0qdIuuNMOdIoAA-5623rQ/videos}{Сергей Глушенко} забросил проект, или сделал его платным с высокой ценой,
и масса пользователей останутся без развития и багфиксов для любимого инструмента}.

Идеология \st\ прямо противоположна\ --- пользователю предоставляется полный доступ ко всем внутренностям системы, вы можете не только
просмотреть исходный код любой системной компоненты до уровня байт-кода, но и внести в них собственные изменения. Возможности
\emph{интерактивной отладки}\ уникальны для \st, ни в одном другом языке вы не можете изменять исходный код, перекомпилировать и подменять
части вашей программы на ходу. \emph{в рантайме}, не останавливая работу программы.

\st\ это штука специфическая\ --- это скорее не язык программирования, и не IDE, а \term{гостевая ОС}\ работающая поверх другой платформы,
со своим языком, архитектурой, и именно этим объясняется ее изолированность, и определенные проблемы интеграции
с другими языками программирования и сервисами \term{хост-системы}.

\begin{itemize}[nosep]
\item интерпретация\note{последние версии CogVM умеют JIT и оптимизации} байт-кода
\item собственная графическая система\note{Squeak и \ph, но некоторые коммерческие системы умеют нативный GUI}
\end{itemize}

\noindent
Чтобы воспользоваться магией \st\ и при этом обойти его изолированность, есть возможность применения
\textit{метапрограммирования через генерацию кода}\ \ref{meta}:
\begin{itemize}[nosep]
\item \emph{модель задачи пишется на \st}\ в виде интерпретируемых структур данных \ref{eds},
\item по модели \emph{синтезируется исходный код на mainstream языках} программирования (\java, \cpp,..).
\item не требуется копировать код между проектами, вместо этого можно иметь общие разделяемымие части между моделями
\item можно смешивать в одной модели описание программы, документацию, и спецификации
\end{itemize}
