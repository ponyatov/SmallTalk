\clearpage
\secrel{System Browser: редактор кода}

\fig{img/browser.png}{width=\textwidth}
\clearpage

\noindent
Наиболее часто используемое окно\ --- просмотр и редактирование классов.\\
По верхнему ряду:
\begin{itemize}[nosep]
    \item пакеты
    \item классы
    \item интерфейсы/категории
    \item методы
    \item нижняя половина: редактор кода
\end{itemize}

\fig{img/br1.png}{width=.55\textwidth}

\noindent
Начнем проект с создания пакета \met\ --- \textit{\emph{meta}programming \emph{Lang}uage}.\\
Откройте браузер \menu{World>Tools>System Browser} или \keys{Ctrl+O+B}, затем \keys{\rms}\ в области пакетов,
\menu{New package>\met>OK}. Затем создадим два новых класса: \keys{\lms}\ на новом пакете, внизу в редакторе кода
измените определение, и \term{подтвердите изменение кода} \menu{\keys{\rms}>Accept} или \keys{Ctrl+S}:

\lst{meta/MetaL.st}{MetaL.st}

Класс \class{MetaL} будет содержать служебные методы и \term{переменные класса}, 
общие для всей системы \met: процедуру инициализации и т.п.

\clearpage
\lst{meta/MFrame.st}{MFrame.st}
\lst{meta/initialize.st}{конструктор \class{MFrame}}

\fig{img/MFrame.png}{width=\textwidth}

Браузер работает в двух режимах, переключаемых под полем категорий: Instance side и Class side. 
При переключании показываются соответственно список методов экземпляра, или самого класса.

\fig{img/install.png}{width=\textwidth}

\lst{meta/install.st}{инициализация \met}
