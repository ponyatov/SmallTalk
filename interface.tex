\secrel{Пользовательский интерфейс}\secdown

\st\ использует свой собственный пользовательский GUI-интер\-фейс, который работает внутри окна виртуальной машины.
Это кажется большим недостатком, так как нативный интерфейс хост-ОС недоступен, как и некоторые сервисы, которые с ним связаны.
Объясняется это тем, что \textit{весь GUI написан на \st}, то есть \emph{вы можете его дополнить или переписать под себя} начиная
с самых низкоуровневых событий, таких как нажатия клавиш, перемещения мыши, и отрисовки графики.

Другое важное достоинство собственного низкоуровневого GUI\ --- интерфейс не зависит от той хост-платформы, на который вы запускаете
виртуальную машину. Не важно, MacOS у вас, \win\ или \linux\note{или даже \href{https://code.google.com/archive/p/squeakvm-tablet/wikis/JenkinsBuilds.wiki}{Android-порт на нетбуке}}\ ---
внешний вид, сочетания клавиш и UX останутся теми же самыми.

\secrel{Выход из системы}

\fig{img/sysexit.png}{height=.5\textheight}

Для выхода из системы откройте самый левый пункт системного меню:

\menu{World>Pharo}

\begin{itemize}[nosep]
    \item \menu{Save and quit} выход с сохранением текущего состояния
    \item \menu{Quit} выход без сохранения: \emph{все измененные данные будут потеряны}, система запустится с последнего сохраненного состояния
\end{itemize}

\secrel{Сохранение состояния системы}

\fig{img/sysexit.png}{height=.5\textheight}

Как было описано в \ref{image}, \st\ сохраняет свое полное состояние в файле образа памяти ВМ\ --- \verb|.image|.
Если вы не хотите потерять текущие данные при внезапном падении системы или ее зависании, в процессе работы нужно сохраняться:
\menu{World>Pharo>Save}\ или \keys{Ctrl+Shift+S}.

\secrel{World: главное меню}

\fig{img/wmenu.png}{height=.45\textheight}

В \st\ рабочий стол системы со всеми графическими объектами называется World (Мир),
соответственно главное (системное) меню называется World Menu. Оно отображается в виде полосы под самым заголовком окна виртуальной машины,
но может быть открыто и как выпадающее меню, если вы щелкните левой кнопкой мыши \keys{\lms}\ на свободном месте рабочего стола.
\menu{World>Pharo}

\secrel{Виды кликов мыши}

\begin{description}
    \item[click левый клик \keys{\lms}]\ \\
    левый короткий клик без дополнительных клавиш-модификаторов. Выбор элемента интерфейса,
    нажатие экранной (графической) кнопки. При \keys{\lms} на свободном месте рабочего стола\ --- открытие \menu{World} меню.
    Нажатая \keys{\lms} с протаскиванием мыши\ --- выделение текста.
    \item[action-click правый клик \keys{\rms}]\ \\
    открытие \term{контекстного меню}\ для элемента интерфейса, который сейчас находится под курсором мыши
    \clearpage\fig{img/halo.png}{height=.6\textheight}
    \item[meta-click мета-клик \keys{Alt+Shift+\mms}] (\linux)\ \\
    нажатие средней кнопки с модификаторами открывает \term{гало-меню}: несколько кнопок по периметру любого элемента GUI,
    которые позволяют изменять его как векторный графический объект\ --- двигать, менять размер, удалить, скопировать,..
\end{description}

\secrel{Элементы заголовка окна}

\fig{img/winmenu.png}{width=\textwidth}

\begin{itemize}
    \item закрыть
    \item минимизировать в низ экрана\ --- будет показываться как прямоугольник с заголовком
    \item открыть на весь экран
    \item заголовок окна
    \item меню окна
\end{itemize}

\secup
